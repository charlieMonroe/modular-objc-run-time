\documentclass[a4paper, 11pt, fleqn]{book}

\usepackage{graphicx}

% Set equal margins on book style
\setlength{\oddsidemargin}{53pt}
\setlength{\evensidemargin}{53pt}
\setlength{\marginparwidth}{57pt}
\setlength{\footskip}{30pt}

\begin{document}

\title{Objective-C Run-time}
\author{Bc. Kry\u{s}tof V\'{a}\u{s}a}
\date{}
\maketitle 

\chapter{Objective-C}
\section{Foreword}

This thesis will analyze source codes of existing versions of Objective-C run-time, their limitations or requirements for compilation. Result of this work will be a prototype of a modular Objective-C run-time, which will allow easy configuration of the run-time environment both at the compilation and run time. For example, for a single-threaded application, you can turn off the locking of internal structures without affecting stability, yet performance can be improved (with each message sent\footnote{In Objective-C method calls are called messages being sent to objects, just like in Smalltalk.}, a lock can be potentially locked when the method implementation isn't cached and the class hierarchy has to be searched) - this may save quite a few syscalls.

There are three available Objective-C run-time implementations (to my knowledge) - one is provided by Apple and is used in its OS X and iOS systems - there are slight differences between the iOS and OS X versions of the run-time (e.g.\ iOS doesn't support garbage collection and only the new 2.0 run-time is available). Within this thesis, when talking about Apple's implementation of the run-time, the OS X version will be the one talked about. Then there's a run-time provided with GCC and a more experimental one called \'{E}toil\'{e} which is used in a GNUStep-based user environment\footnote{http://www.jot.fm/issues/issue\_2009\_01/article4/index.html}.

Even though I will mention a few words about the garbage collection and ARC\footnote{ARC - automatic reference counting, a feature introduced in Xcode 4.2 (Xcode is Apple's IDE) that uses compiler's static analysis combined with special keywords to automatically insert retain/release calls so that the developer doesn't need to manually manage reference counts on objects.}, not much attention will be paid to them as garbage collection is being deprecated in OS X 10.8 (TODO - ELABORATE IN APPLE SECTION - and has severe dependencies on Mac OS X itself) and ARC is relatively new and uses a lot of compiler-dependent features as well as auto-zeroing weak references, etc.; which is beyond the scope of this work. Instead, the focus will be put on the core functionality of the run-time, analysis of the current implementations and designing the modular run-time itself.
  
\section{Brief history of Objective-C}

In the early 1980s, Brad Cox and Tom Love decided to bring the object-oriented concept to the world of C while maintaining full backward compatibility, strongly inspired by Smalltalk

In 1988, NeXT has licensed Objective-C from Stepstone (the company Cox and Love owned), added Objective-C support to the GCC compiler and decided to use it in its OpenStep and NeXTStep operating systems.

After Apple had acquired NeXT in 1996, Objective-C stayed alive in Rhapsody\footnote{http://en.wikipedia.org/wiki/Rhapsody\_(operating\_system)} and later on in Mac OS X, where it's the preferred programming language to the date.

For this whole time, the Objective-C language stayed almost the same without any significant changes. In 2006, Apple announced Objective-C 2.0 (which was released in Mac OS X 10.5 in 2007), which introduced garbage collection (which has been deprecated in 10.8 in favor of more efficient ARC - automatic reference counting\footnote{http://cocoaheads.tumblr.com/post/17719985728/10-8-objective-c-enhancements}), properties (object variables with automatically generated getters and/or setters with specified memory management), fast enumeration (enumeration over collections in a foreach-style), and some other minor improvements.

Lately, more improvements have been made to Objective-C, most importantly the aforementioned ARC (automatic reference counting). Apple's run-time a hardcoded set of selectors (method names) that handle the memory management, -autorelease, -retain, -release (together called ARR), in particular. ARC automatically inserts these method calls and automatically generates a -dealloc method (which is called when the object is being deallocated) - which requires compiler support, though.

This, however, presents a problem - none of the ARR calls must be called directly in the code - hence you need to convert all of your code to ARC\@. One disadvantage which results in a big advantage - compatibility with all libraries (Apple calls Objective-C libraries frameworks) - this was a big disadvantage of garbage collection: 

You could keep the code as it was as the run-time itself redirected the ARR methods to a no-op function on the fly, however, all linked libraries/frameworks/plugins needed to be recompiled with garbage collection support turned on. This caused two things: mess in the code as if you migrated your code to garbage-collection-enabled environment, it was riddled with ARR calls, however, newly written code typically omitted those calls, making the code inconsistent; and some libraries never got GC support anyway, so you couldn't use them in GC-enabled applications.

In the newest release of OS X 10.8, several new features have been included - default synthesis of getters (in prior versions, you had to declare \verb=@property= in the header file and use \verb=@sythesize= or \verb=@dynamic= in the implementation file - see Syntax of Objective-C), type-safe enums, literals for NSArray, NSDictionary and NSNumber (classes declared in Apple's Foundation framework), etc.

\section{Compilation of Objective-C}

Objective-C is an object-oriented programming language that is a strict superset of C. Any C code can be used within Objective-C source code. Its run-time is written in C as well, some parts in assembly language (mostly performance optimizations) or more recently in C++ (more about that later on). This thesis assumes that you have some brief knowledge of both C and Objective-C, at least syntax-wise.

All Objective-C code can be translated to calls of C run-time functions\footnote{There's a LLVM Clang compiler option -rewrite-objc which will convert all the Objective-C syntax into calls of pure C methods - the run-time methods. When run `clang -rewrite-objc test.m', where test.m contains the Objective-C code, a new test.cpp is created, containing the translated code.} - for example, sending a message to an object isn't anything else than calling a run-time function \verb=objc_msgSend=.

This section will cover compilation of Objective-C code and how it's translated into calling run-time functions.

\subsection{Calling methods}

This is a sample code that sends two messages - each to a different object, though\footnote{As will be explained later on, each class actually consists of two classes - the meta class, which has the +alloc method and the regular class, which has the -init method.}:
\begin{verbatim}SomeClass *myObj = [[SomeClass alloc] init];\end{verbatim}

This will be translated to:
\begin{verbatim}SomeClass *myObj = ((id (*)(id, SEL, ...))(void *)objc_msgSend)
              ((id)((id (*)(id, SEL, ...))(void *)objc_msgSend)
                                  (objc_getClass("SomeClass"),
                                  sel_registerName("alloc")), 
                                  sel_registerName("init"));
\end{verbatim}

Which after removing the casting and adding a little formatting is:

\begin{verbatim}SomeClass *myObj = 
  objc_msgSend(
    objc_msgSend(
      objc_getClass("SomeClass"),  
      sel_registerName("alloc")), 
    sel_registerName("init"));
\end{verbatim}

So it's two nested \verb=objc_msgSend= calls\footnote{There are actually specific functions for methods that return floating point numbers or structures, as these may require special ABI treatment on some architectures.}. \verb=objc_msgSend= is a method that can be said to be the core of Objective-C run-time. It's the most used function of the run-time. Every method call in Objective-C gets translated into this variadic function call, which takes \verb=self= as the first argument (i.e.\ the object that the method is called on, or the message is sent to\footnote{Here can be seen the Smalltalk influence.}), the second argument is a selector (generally the method's name) and can be followed by arguments.

The run-time then looks up the object's class, finds a function that implements this method (so called \verb=IMP=\footnote{IMP is defined as a pointer to a function: id (*IMP)(id, SEL, ...).}) and calls it. There's a several things to point out:

• method \emph{names} are used. \verb=sel_registerName= is a function that makes sure that for that particular method name only one selector pointer is kept.
• each of the calls goes to a different object. The first call gets to something returned by \verb=objc_getClass= which returns an instance of a meta class (which is an object as well)
• every class consists of two classes - a class pair - one regular of which you create objects and one meta - which typically (unless you manually craft another one) has only one instance: a receiver for class methods (static methods).

If you care to investigate this hierarchy, here's a small example:

\verb=NSObject= class is part of the Foundation framework Apple supplies. Even though its often assumed to be the one and only root class in Objective-C, this is quite incorrect: there can be as many root classes in Objective-C as you wish - \verb=NSProxy= is an example and you can easily create your own.

\begin{verbatim}@interface ClassWithoutSuperclass
@end\end{verbatim}

This will declare a new root class. It has absolutely no functionality - no memory management \verb=retain= and \verb=release=, no \verb=+alloc= method is declared either - you wouldn't be able to even create a new instance of this class without the run-time function \verb=class_createInstance=\footnote{This is basically why it is recommended to inherit all classes from NSObject (or any other root class) which implements some basic communication with the run-time as well as some basic memory management, etc.}.

Each object is a pointer to a structure. Let's use this simplified class structure (it's more complex in real life):

\begin{verbatim}
typedef struct class_t {
  struct class_t *isa;
  struct class_t *superclass;
} objc_class_t;
\end{verbatim}

Every object is a pointer to a structure like this one. The first field, so-called \verb=isa=, is a pointer to the class structure (to the meta class instance). Second field points to the superclass. It might be confusing at first that it's a class structure, but remember that even the object's class is actually an object - an instance of the meta class\footnote{The root meta class's isa pointer points to the structure itself - i.e. there's a pointer cycle.}.

Assume the following code:
\begin{verbatim}@interface Rootclass
@end
@implementation Rootclass
@end

@interface Subclass : Rootclass
@end
@implementation Subclass
@end 
\end{verbatim}

This declares two classes - \verb=Rootclass= and \verb=Subclass=. The \verb=Rootclass= is a new root class with no superclass. As neither of these classes declares any methods, calling anything on either class would result in a run-time exception, even the usual object creation via \verb=[[Rootclass alloc] init]= isn't available as \verb=Rootclass= doesn't declare the \verb=+alloc= method - it's declared on the \verb=NSObject= class, which is why you can create instances of the ``regular" classes inheriting from \verb=NSObject= this way.

Hence we need to use the run-time \verb=class_createInstance= function to create an instance of the class: 
\begin{verbatim}
id obj = class_createInstance(objc_getClass("Subclass"), 0);
\end{verbatim}

The \verb=objc_getClass= function returns a pointer to the class called \verb=Subclass=, the \verb=objc_getClass= function creates an instance of the \verb=Subclass= class, with \verb=0= extra bytes\footnote{TODO: explain extra bytes.}. Here's a class and meta-class diagram of this situation.

\includegraphics[width=120mm]{metaclass_graph.png}

\subsection{Creating Classes Programmatically}

There is a function called \verb=objc_allocateClassPair= which creates a brand new class and its meta class (together a class pair) on the run. All you need to specify is the superclass, new class name and extra bytes. Using functions such as \verb=class_addMethod=, \verb=class_addIvar=, \verb=class_addProtocol= and \verb=class_addProperty= can be used to add methods, ivars, protocols and properties to a class.

Using these methods, you can easily substitute the Objective-C compiler, creating all classes at the beginning of the application run. 

In reality, declaring a class doesn't cause the compiler to generate function calls, however, instead, the compiler creates static class structures which are later on copied by the linker into the \verb=__OBJC= section of the Mach-O binary (on OS X), which is copied on the launch time to memory and the classes just get registered to the run-time\footnote{Creating the class using the objc\_allocateClassPair function isn't enough in order to create an instance of this class, you need to register the class pair with the run-time as well using objc\_registerClassPair. This is simply to avoid creating an instance of the class before it gets fully initialized, e.g.\ from a different thread.}, which is much faster that dynamically create classes one by one, connecting all methods. We will, however, focus on the run-time methods, ignoring linker dependencies.

\subsection{Translating Methods to Functions}

As noted several times before, all Objective-C code can be rewritten in pure C code. This brings us to a question, how are the methods translated to C constructs - obviously into a function, so-called \verb=IMP=. Let's use this class to demonstrate:

\begin{verbatim}

@implementation SomeClass
+(void)doSomethingStatic{
  // ...
}
-(void)someMethod:(void*)arg1 secondArgument:(void*)arg2{
  // ...
}
@end

\end{verbatim}

This gets translated into two functions:

\begin{verbatim}
typedef struct objc_object SomeClass;

void _C_SomeClass_doSomethingStatic(Class self, SEL _cmd){
  // ...
}

void _I_SomeClass_someMethod_secondArgument(SomeClass *self,
                             SEL _cmd, void *arg1, void *arg2){
  //...
}
\end{verbatim}

As you can see, each method gets translated into a function of at least two parameters. The first argument is \verb=self= - a pointer to the object the message is being sent to. In the first case a \verb=Class= object\footnote{The Class is defined as typedef struct objc\_class *Class.}, in the second case a pointer to the \verb=SomeClass= object. The second argument, \verb=_cmd=, is the selector (\verb=SEL=). Selector is a structure that consists of just the method's name, so theoretically, it's possible to simply retype \verb=char*= to \verb=SEL=, however, shouldn't be done as the run-time requires equally named selectors to point to the same structure (that's why the \verb=sel_registerName= should be used to convert a \verb=char*= to \verb=SEL= which unifies the equally named selectors).

The function names get slightly obfuscated - \verb=_X_ClassName_method_name_= - where \verb=X= is either \verb=I= for instance methods or \verb=C= for class methods. As Objective-C method names can have multiple parts, each followed by a semi-colon (e.g.\ \verb=someMethod:secondArgument:=), each part gets concatenated using an underscore.

\chapter{Apple's Implementation}
\section{Limitations}
// TODO
- Bridging, 16B objects minimum.

\end{document}
      