\chapter*{Introduction}
\addcontentsline{toc}{chapter}{Introduction}

Every developer probably has a favorite programming language they feel the most comfortable writing code in. Mine has been Objective-C for a long time now, which made me wonder about the technical background of the language.

I asked myself how does it really work - what is the path from a source code file to a running application. In real life all paths are supposed to lead to Rome. In my case, all paths led to something called \emph{Objective-C run-time}.

While there already are existing Objective-C run-times, a closer look at them shows their weaknesses - mostly their dependencies or assumptions about the underlying OS and environment, in which they are supposed to be running. Be it the POSIX layer, minimal object size, or GCC-specific C language extensions, all of these dependencies may present an obstacle when trying to compile such a run-time for a non-typical operating system (for example an experimental one), or when trying to use the Objective-C run-time in the kernel code.

This thesis analyzes source codes of existing versions of Objective-C run-time, their limitations or requirements for compilation. Result of this work is a prototype of a very flexible run-time in terms of modularity - the run-time is very bare feature-wise, yet offers ways to be easily extended. For example, the run-time includes no support for categories, yet a class extension is included which lets you add support for it. Also, the run-time allows easy configurability - for instance, a single-threaded application is able to turn off locking of internal structures without affecting stability, yet performance can be improved (with each message sent\footnote{In Objective-C method calls are called messages being sent to objects, just like in Smalltalk.}, a lock can be potentially locked when the method implementation is not cached). This may save quite a few syscalls.

The resulting run-time has virtually zero dependencies on the underlying operating system as well as the compiler. The whole run-time is written to be C89 compatible and as a proof of its flexibility, the run-time has not only been run on common systems such as OS X, Windows XP and Linux. It has also been successfully run on more 'exotic' systems, such as Windows 3.11 and Kalisto HeSiVa, an experimental operating system written by me and my colleagues Jiří Helmich and Jan Široký for an Operating Systems course at school.
