\chapter{Objective-C}
  
In the early 1980s, Brad Cox and Tom Love decided to bring the object-oriented concept to the world of C while maintaining full backward compatibility. The result was Objective-C, a language heavily inspired by Smalltalk. At first, the language had no compiler support, but as all Objective-C code can be actually rewritten in pure C, a preprocessor was a sufficient tool.

In 1988, NeXT has licensed Objective-C from Stepstone (the company Cox and Love owned), added Objective-C support to the GCC compiler and decided to use it in its OpenStep and NeXTSTEP operating systems (many classes in Apple's frameworks have a \verb=NS= prefix to the date, which stands either for NeXTSTEP, or NeXT-Sun as the OpenStep operating system had been developed in cooperation with Sun Microsystems\footnote{http://en.wikipedia.org/wiki/OpenStep}).

After Apple had acquired NeXT in 1996, Objective-C stayed alive in Rhapsody\footnote{http://en.wikipedia.org/wiki/Rhapsody\_(operating\_system)} and later on in Mac OS X, where it is the preferred programming language to the date.

For this whole time, the Objective-C language stayed almost the same without any significant changes, until 2006, when Apple announced Objective-C 2.0 (released as a part of Mac OS X 10.5 in 2007), which introduced garbage collection (since then deprecated in 10.8 in favor of more efficient ARC - automatic reference counting\footnote{http://cocoaheads.tumblr.com/post/17719985728/10-8-objective-c-enhancements}), properties (object variables with automatically generated getters and/or setters with specified memory management), fast enumeration (enumeration over collections in a foreach-style), and some other minor improvements.

Lately, more improvements have been made to Objective-C, most importantly the aforementioned ARC (automatic reference counting). Apple's run-time has a hardcoded set of selectors (method names) that handle the memory management, \verb=-autorelease=, \verb=-retain=, \verb=-release= (together called ARR), in particular. ARC automatically inserts these method calls and automatically generates a \verb=-dealloc= method (which is called when the object is being deallocated) - this, however, adds a heavy dependency on the compiler, though, as it needs to statically analyze the code in order to safely insert these ARR calls.

This, however, presents a problem - none of the ARR calls must be called directly in the code - hence all code needs to be converted to ARC\@. One disadvantage which results in a big advantage - compatibility with all existing frameworks. The backward incompatibility was a big disadvantage of garbage collection: 

The code could be kept as it was - the run-time itself redirects the ARR methods to a no-op function on the fly. All linked libraries / frameworks / plugins, however, needed to be recompiled with garbage collection support turned on. This caused two things: mess in the code as the code migrated to garbage-collection-enabled environment was riddled with ARR calls, but newly written code typically missed those calls, making the code inconsistent. Also, some libraries never got GC support anyway, so they couldn't be used in GC-enabled applications.

In the newest release of OS X - 10.8, several more new features have been included - default synthesis of getters (in prior versions, you had to declare \verb=@property= in the header file and use \verb=@sythesize= or \verb=@dynamic= in the implementation file), type-safe enums, literals for \verb=NSArray=, \verb=NSDictionary= and \verb=NSNumber= (classes declared in Apple's Foundation framework), etc.

\section{Objective-C Syntax in a Nutshell}

While a complete reference to the language is not a goal of this work, a quick syntax overview is included for readers without any Objective-C knowledge.

\subsection{Class Declaration}

A class can be declared using the following notation:

\begin{figure}[H]
  \begin{verbatim}
@interface Class : Superclass {
  int anIvar;
}

// Method and property declarations follow:

+(void)classMethod;

-(int)instanceMethod;

@property BOOL myProperty;

@end
  \end{verbatim}
  \centering{}
  \caption{Declaration of an Objective-C class.}
  \label{fig:syntax_objc_class_decl}
\end{figure}

The class declaration may be divided into two sections:

\begin{itemize}
  \item{\bf{Ivars}} A list of ivars, which can be completely omitted if the class has no ivars to declare.
  \item{\bf{Method and property declarations}} Methods and properties may be declared here. Methods that start with the \verb=+= sign are class (static) methods, methods starting with the \verb=-= sign are instance methods.
\end{itemize}

After declaring a class, it needs to be implemented as well:

\begin{figure}[H]
  \begin{verbatim}
@implementation Class

+(void)classMethod{
  // Method body goes here
}

-(int)instanceMethod{
  // Method body goes here
}

@end
  \end{verbatim}
  \centering{}
  \caption{Implementation of an Objective-C class.}
  \label{fig:syntax_objc_class_impl}
\end{figure}

In earlier versions of the run-time (and OS), each property had to be matched with a \verb=@synthesize= construct in the implementation part of the class, which is no longer necessary as it gets generated automatically.

\subsection{Declaring Methods}

As has been seen in figure ~\ref{fig:syntax_objc_class_decl}, each method declaration begins with either a \verb=+= sign (class method), or \verb=-= sign (instance method), followed by a return type, method name and potentially arguments.

If the method should have an argument, it is followed by a semicolon, type of the argument and the argument name:

\begin{figure}[H]
  \begin{verbatim}
-(void)setName:(NSString*)name;
  \end{verbatim}
  \centering{}
  \caption{Method declaration with a single argument.}
  \label{fig:syntax_objc_single_arg}
\end{figure}

More arguments are allowed as well, splitting the method name:

\begin{figure}[H]
  \begin{verbatim}
-(void)writeToFileAtURL:(NSURL*)url 
              usingEncoding:(NSStringEncoding)enc;
  \end{verbatim}
  \centering{}
  \caption{Method declaration with multiple arguments.}
  \label{fig:syntax_objc_multiple_args}
\end{figure}

\subsection{Calling Methods}

Calling methods, or in Smalltalk's terminology sending messages, is achieved using the following formulations:

\begin{figure}[H]
  \begin{verbatim}
[Class classMethod];
[obj instanceMethod];
[obj setName:@"John"];
[obj writeToFileAtURL:url usingEncoding:NSUTF8StringEncoding];
  \end{verbatim}
  \centering{}
  \caption{Calling methods on a class and an object.}
  \label{fig:syntax_objc_msg_sending}
\end{figure}

\subsection{Miscellaneous}

\begin{itemize}
  \item{\bf{Strings}} As regular C strings cannot be treated as an object, a special notation is introduced for Objective-C strings - \verb=@"String"=.
  \item{\bf{Synchronization}} A special synchronization construct \newline{}\verb=@synchronized(obj){ ... }= can be used for critical section code.
  \item{\bf{Hidden variables}} In each method, three special variables may be accessed: \verb=self= (pointing to the object itself), \verb=_cmd= (the method name) and \verb=super= (allowing invocation of the superclass' implementation of methods).
\end{itemize}


\section{Compilation of Objective-C}

Objective-C is an object-oriented programming language that is a strict superset of C. Any C code can be used within Objective-C source code. Its run-time is written in C as well, some parts in the assembly language (mostly performance optimizations) or more recently in C++ (more about that later on).

And the other way around, all Objective-C code can be translated to calls of C run-time functions. There is an LLVM Clang compiler option \verb=-rewrite-objc= which converts all the Objective-C syntax into calls of pure C methods - the run-time methods. When run 

\begin{verbatim}
  clang -rewrite-objc test.m
\end{verbatim}

where \verb=test.m= contains the Objective-C code, a new test.cpp is created, containing the translated code. For example, sending a message to an object is nothing else but calling a run-time function \verb=objc_msgSend=. Note that the run-time implementations may differ in the function names, or even use different structures and the way methods are invoked. The examples below will show how the code translation works with Apple's version of the run-time. These examples are here to simply illustrate the mechanism of translating Objective-C code to C constructs.

\subsection{Calling methods}

As has been mentioned before, calling a method is not anything else than calling a \verb=objc_msgSend= variadic function, which is responsible for finding a function pointer that implements the actual method and invoking it, passing all the arguments along

\paragraph{Example}
Here is a sample code that sends two messages - each to a different object, though - each class actually consists of two classes - the meta class, which implements the \verb=+alloc= method and the regular class (an instance of the metaclass), which implements the \verb=-init= method.

\begin{verbatim}SomeClass *myObj = [[SomeClass alloc] init];\end{verbatim}

This gets to be translated to:
\begin{verbatim}SomeClass *myObj = ((id (*)(id, SEL, ...))(void *)objc_msgSend)
              ((id)((id (*)(id, SEL, ...))(void *)objc_msgSend)
                                  (objc_getClass("SomeClass"),
                                  sel_registerName("alloc")), 
                                  sel_registerName("init"));
\end{verbatim}

Which after removing the casting and adding a little formatting is equivalent with:

\begin{figure}[H]
  \begin{verbatim}SomeClass *myObj = 
  objc_msgSend(
    objc_msgSend(
      objc_getClass("SomeClass"),  
      sel_registerName("alloc")), 
    sel_registerName("init"));
  \end{verbatim}
  \centering{}
  \caption{Objective-C message calls translated to run-time function calls.}
  \label{fig:methods_translated_to_objcMsgSend}
\end{figure}

So it is two nested \verb=objc_msgSend= calls. There are actually specific functions for methods that return floating point numbers or structures, as these require special ABI treatment on some architectures - for example structures that do not fit into registers are returned by reference on the stack as a hidden first argument of the function. \verb=objc_msgSend= is a function that can be called the core of Objective-C run-time. It is the most used function of the run-time. Every method call in Objective-C gets translated into this variadic function call, which takes \verb=self= as the first argument (i.e.\ the object the message is sent to), the second argument is a selector (generally the method's name) and can be followed by any number arguments.

\subparagraph{GCC Run-time}

The GCC run-time differs slightly from Apple's run-time - it does not have a \verb=objc_msgSend= function, but uses \verb=objc_msg_lookup= function, which rather returns a pointer to the implementation function itself (the so-called \verb=IMP=). The same example would compile under the GCC run-time into the following calls:

\begin{figure}[H]
\begin{verbatim}
  id receiver1 = objc_getClass("SomeClass");
  SEL selector1 = sel_registerName("alloc");
  IMP allocIMP = objc_msg_lookup(receiver1, 
                                  selector1);
  id receiver2 = allocIMP(receiver1, selector1);
  SEL selector2 = sel_registerName("init");
  IMP initIMP = objc_msg_lookup(receiver2, 
                                  selector2);
  SomeClass *myObj = initIMP(receiver2, selector2);
\end{verbatim}
\centering{}
  \caption{Objective-C message calls translated to run-time function calls in GCC run-time.}
  \label{fig:methods_translated_to_objc_msg_lookup}
\end{figure}

As you can see, Apple's run-time differs only in the fact that the \verb=objc_msgSend= calls the function directly, whereas the GCC run-time looks up the function and then calls it. This has a small advantage that it does not require any special variants of functions, like in Apple's run-time, where the \verb=objc_msgSend= has 4 variants depending on the return type.

\hspace{20pt}

But even so, the principe is the same as in Apple's run-time. The run-time needs to look up the object's class, find a function that implements that particular method (the so called \verb=IMP=) and the function gets called either by \verb=objc_msgSend=, or directly. There are several things to point out:

\begin{itemize}
\item method \emph{names} are used. \verb=sel_registerName= is a function that makes sure that for that particular method name only one selector pointer is kept. A selector is a pointer to a structure representing a method name. On some run-times, selectors are typed, i.e.\ methods of the same name, but with different argument types result in different selectors. While Objective-C does not support method overloading, the selector storage is program-wise, not just per class.
\item every class consists of two classes - a class pair - one regular of which you create objects and one meta - which typically (unless you manually craft another one) has only one instance: a receiver for class methods (static methods).
\item each of the calls is sent to a different object. The first call is sent to something returned by \verb=objc_getClass= which returns a class, which is an instance of its meta class (which is an object as well). The second call goes already to an object - instance of the class (the class hierarchy is described below).
\end{itemize}

\paragraph{Calls to super}

Objective-C, as most object-oriented languages, allows calls of the method implementations of a superclass using the keyword \verb=super=. For example, the \verb=-init= method usually starts with \verb#if ((self = [super init]) != nil)#. This is done using a special structure \verb=objc_super=, which is passed by reference to \verb=objc_msgSendSuper= (or its relatives) in case of Apple run-time, or \newline{}\verb=objc_msg_lookup_super= in case of GCC run-time.

\subparagraph{Example}

\begin{verbatim}
-(void)someMethod:(void*)firstArgument{
  [super someMethod:firstArgument];
}
\end{verbatim}

Is equivalent to:

\begin{figure}[H]
\begin{verbatim}
-(void)someMethod:(void*)firstArgument{
  struct objc_super super = { 
      self, 
      class_getSuperclass(objc_getClass("SomeSubclass")) 
  };
  objc_msgSendSuper(&super, 
                    sel_registerName("someMethod:"), 
                    firstArgument);
}
\end{verbatim}
\centering{}
  \caption{Objective-C message call to super translated to a run-time function call.}
  \label{fig:methods_translated_to_objcMsgSendSuper}
\end{figure}

\subsection{Object Model}

As Objective-C is heavily influenced by Smalltalk, let us examine the Smalltalk object model first\footnote{All copyright for this image goes to Andrew P. Black St\'{e}phane Ducasse, Oscar Nierstrasz Damien Pollet and Damien Cassou and Marcus Denker - http://pharo.gforge.inria.fr/PBE1/PBE1.html}:

\begin{figure}[H]
\includegraphics[width=120mm]{./img/smalltalk_class_hierarchy.png}
  \centering{}
  \caption{Smalltalk's object model.}
  \label{fig:smalltalk_obj_model}
\end{figure}

The model may seem complex and even a little confusing in some areas. Smalltalk's approach that everything is an instance of some class poses a question where is the end to the class hierarchy? Most languages that introduce a concept of a metaclass have to solve this by a loop: in Smalltalk's case, it is the loop between \verb=Metaclass= and \verb=Metaclass class=, where each is an instance of the other. The Objective-C object model indeed has a similar loop, however. It is not that complex and would end at the \verb=Object class= point in the Smalltalk diagram.

\paragraph{Root Class}
The \verb=NSObject= class is part of the Foundation framework Apple supplies. While it is often assumed to be the one and only root class in Objective-C, this is quite incorrect: there can be as many root classes in Objective-C as you wish - \verb=NSProxy= is an example and you can easily create your own:

\begin{figure}[H]
\begin{verbatim}@interface ClassWithoutSuperclass {
  
}

@end\end{verbatim}
  \centering{}
  \caption{Declaring a root class in Objective-C.}
  \label{fig:objc_root_class}
\end{figure}

This declares a new root class. It has absolutely no functionality - no memory management methods such as \verb=-retain= and \verb=-release=, no \verb=+alloc= method is declared either - it is impossible to even create a new instance of this class without the run-time function \verb=class_createInstance= - which is basically why it is recommended to inherit all classes from NSObject (or any other already prepared root class) which implements some basic communication with the run-time as well as some basic memory management, etc. Also, Apple's run-time has hardcoded references to \verb=NSObject=, which enables faster ARR message dispatch (as it checks if the class has any custom ARR-method implementation).

In Objective-C, each object is a pointer to a structure, where the first member is a so-called \verb=isa= pointer. Actually, the \verb=id= type, that represents any Objective-C object is defined as follows:

\begin{figure}[H]
\begin{verbatim}
typedef struct objc_class *Class;
typedef struct {
  Class isa;
} *id;
\end{verbatim}
  \centering{}
  \caption{Objective-C's object definition.}
  \label{fig:objc_obj_def}
\end{figure}

To support the behavior that a class (\verb=Class=) is an object as well, the class structure begins with the \verb=isa= pointer, which points to its meta class. The \verb=isa= pointer is followed by a \verb=superclass= field and many others - ivars, methods, cache, dispatch table, etc. - depending also on the run-time that you are using - you should hence never rely on the class structure itself, it should be treated as an opaque structure, with the internals exposed only to the run-time itself. To simplify this example, let us use this simplified class structure:

\begin{figure}[H]
\begin{verbatim}
typedef struct class_t {
  struct class_t *isa;
  struct class_t *superclass;
  
  // Actually, more fields follow
} objc_class_t;
\end{verbatim}
  \centering{}
  \caption{Simplified structure representing an Objective-C class.}
  \label{fig:objc_class_struct_simplified}
\end{figure}

For a regular class, the \verb=isa= pointer points to its meta class and the \verb=superclass= pointer points to its regular superclass, or \verb=Nil= in case it is the root class (in Objective-C, the zero pointer is called \verb=nil= for objects and \verb=Nil= for classes - both are just \verb=#define=s of a typed zero, though). 

Now how about the meta class? In case the class is not a root class, the \verb=superclass= pointer points to its meta superclass and the \verb=isa= pointer points to the same meta class.

Hence the regular class is an instance of its superclass. In case the class is a root class, its \verb=isa= pointer points to the structure itself, and its superclass is the regular class. Let's examine this on an example:

\begin{verbatim}@interface Rootclass{
  
}

@end

@implementation Rootclass

// Empty implementation

@end



@interface Subclass : Rootclass{

}

@end

@implementation Subclass

// Empty implementation

@end 
\end{verbatim}

This declares two classes (actually four, as for each class a meta class is created as well) - \verb=Rootclass= and \verb=Subclass=. The \verb=Rootclass= is a new root class with no superclass. As neither of these classes declares any methods, calling anything on either class would result in a run-time exception, even the usual object creation via \verb=[[Rootclass alloc] init]= is not available as \verb=Rootclass= does not declare the \verb=+alloc= method - it is declared on the \verb=NSObject= class, which is why you can create instances of the ``regular" classes inheriting from \verb=NSObject= this way.

Hence the run-time function \verb=class_createInstance= needs to be called in order to create an instance of the class:

\begin{verbatim}
id obj = class_createInstance(objc_getClass("Subclass"), 0);
\end{verbatim}

The \verb=objc_getClass= function returns a pointer to the class called \verb=Subclass=, the \verb=class_createInstance= function creates an instance of the \verb=Subclass= class, with \verb=0= extra bytes - the extra bytes parameter is here in case you wanted to dynamically add extra space to some of your instances, or to all by overriding the \verb=+alloc= method of the class. It is noteworthy that the \verb=+alloc= method is actually nothing else than \verb=-alloc= on the meta class - i.e.\ all class methods are in fact instance methods on the meta class.


Diagram on figure ~\ref{fig:class_metaclass_graph} visualizes this situation of two classes - one root class and its subclass; and an instance of the subclass.

\begin{figure}[H] 
\includegraphics[width=\textwidth]{img/metaclass_graph.png}
  \centering{}
  \caption{A graph of the class - meta-class relationship.}
  \label{fig:class_metaclass_graph}
\end{figure}

It is obvious from the diagram that the Objective-C object model is far less complicated than Smalltalk's. The class hierarchy ends at the point of the root class. To sum it up:

\begin{itemize}
\item Each class consists actually of two classes: the regular class and the meta class.
\item As the class hierarchy goes, the meta class hierarchy follows the regular class hierarchy up to the root class.
\item It is easy to detect a meta class - \verb:cl->superclass == cl->isa:.
\item Since the regular class is just an instance of the meta class, having a pointer to the meta class, there is no way of retrieving a pointer to the regular class. This is why the run-times store pointers to the regular classes and not their meta classes.
\item The root class is a little special, as the meta class is an instance of itself and its superclass is the regular class (which is an instance of the meta class).
\item This has a peculiar consequence: all instance methods of the root class are class methods as well. In the lookup chain, when the search on meta classes yields nothing, the run-time follows the \verb=superclass= pointer, which points to the regular class object. This can be easily verified:
\end{itemize}

\begin{figure}[H]
\begin{verbatim}
unsigned int number_of_methods = 0;
Method *methods, *method_ptr;
methods = method_ptr = class_copyMethodList([NSObject class], 
                                            &number_of_methods);
while (number_of_methods){
  printf("%s\n", sel_getName(method_getName(*methods)));
  ++methods;
  --number_of_methods;
}
free(method_ptr);
\end{verbatim}
  \centering{}
  \caption{Listing methods of \texttt{NSObject} class.}
  \label{fig:listing_NSObject_methods}
\end{figure}

This piece of code prints all available methods on a class. The \verb=[NSObject= \verb=class]=, however, is the regular class, which on the installation of OS X it has been tested, prints out 298 methods. Note that these are methods directly implemented by that class. Methods implemented by the class' superclasses are not included.

When \verb=[NSObject class]= is replaced with

\begin{verbatim}
methods = method_ptr = class_copyMethodList(
                                ((id)[NSObject class])->isa, 
                                &number_of_methods
                                           );
\end{verbatim}

a list of methods declared directly on the meta class is printed. This list counts only 118 methods, among which, for example, is not a method \verb=isNSArray__=, which is a private \verb=NSObject= instance method for deciding whether the object is an instance of \verb=NSArray= class. While the meta class itself does not implement this method, however, calling

\begin{verbatim}
[((id)[NSObject class])->isa isNSArray__];
\end{verbatim}

does not yield in any run-time exception, or similar, it simply invokes the instance method. This can be further proved by exchanging the function pointer of \verb=NSObject='s \verb=isNSArray__= method with a custom function, that prints a message, for example.

While this trickery might be viewed as unnecessary and almost 'insane', it has a good reasoning behind - it is mostly because of this that you can treat classes as regular objects without implementing the same methods twice - once as instance methods and once as class methods.

\subsection{Creating Classes Programmatically}

While this is not a feature of the run-time that would be used on a daily basis, classes can be created and inserted into the run-time at any time. There is a function called \verb=objc_allocateClassPair= which creates a brand new class and its meta class counterpart - together a class pair. All that is needed to be specified is the superclass, the new class' name and extra bytes. These extra bytes are similar to the extra bytes argument of \verb=objc_createInstance=, but this time they represent extra bytes on the class structure itself for possible class functionality extension. Run-time functions such as \verb=class_addMethod=, \verb=class_addIvar=, \verb=class_addProtocol= and \verb=class_addProperty= can be used to add methods, ivars, protocols and properties to a class.

Creating the class using the \verb=objc_allocateClassPair= function is not enough in order to create an instance of this class, though. It is necessary to register the class pair with the run-time as well, using \verb=objc_registerClassPair=. This is simply to avoid creating an instance of the class before it gets fully initialized, e.g.\ from a different thread. Using these functions, the Objective-C compiler may easily be substituted, creating all classes at the beginning of the program execution programatically.

In reality, declaring a class does not cause the compiler to generate function calls. Instead, the compiler creates static class structures which are later on copied by the linker into the \verb=__OBJC= section of the Mach-O binary (on OS X), which is copied on the launch time to memory and the classes just get registered with the run-time (the dynamic loader calls some private run-time methods for copying classes from binary images), which is much faster than dynamically creating classes one by one, connecting all methods, ivars, etc. This thesis, however, focuses on the run-time methods, ignoring linker and dynamic loader dependencies.

\subsection{Translating Methods to Functions}

As noted several times before, all Objective-C code can be rewritten in pure C code. This brings us to a question, how the methods are translated to C constructs - obviously into functions, so-called \verb=IMP=s (an implementation pointer). Let us use this class to demonstrate:

\begin{verbatim}

@implementation SomeClass
+(void)doSomethingStatic{
  // ...
}
-(void)someMethod:(void*)arg1 secondArgument:(void*)arg2{
  // ...
}
@end

\end{verbatim}

This gets translated into two functions:

\begin{figure}[H]
\begin{verbatim}
typedef struct objc_object SomeClass;

void _C_SomeClass_doSomethingStatic(Class self, SEL _cmd){
  // ...
}

void _I_SomeClass_someMethod_secondArgument(SomeClass *self,
                             SEL _cmd, void *arg1, void *arg2){
  //...
}
\end{verbatim}
  \centering{}
  \caption{Methods translated into C functions.}
  \label{fig:methods_translated_into_C_fncts}
\end{figure}

As can be seen, each method gets translated into a function of at least two arguments. The first argument is \verb=self= - a pointer to the object the message is being sent to. In the first case a \verb=Class= object, in the second case a pointer to the \verb=SomeClass= object. The second argument, \verb=_cmd=, is the selector (\verb=SEL=). Using a function with the same signature, method implementations of existing methods may be easily exchanged.

The function names get slightly obfuscated - \verb=_X_ClassName_method_name_= - where \verb=X= is either \verb=I= for instance methods or \verb=C= for class methods. As Objective-C method names can have multiple parts, each followed by a semi-colon (e.g.\ \verb=someMethod:secondArgument:=), each part gets concatenated using an underscore.

\subsection{Synchronization and Exceptions}

Objective-C has a \verb=@synchronized(obj)= syntax, which locks a recursive mutex associated with \verb=obj= at the beginning of the synchronization scope and unlocks it at the end. For example:

\begin{verbatim}
...

@synchronized(self){
  // Critical code goes here
}
...
\end{verbatim}

gets translated into:

\begin{figure}[H]
\begin{verbatim}
...
objc_sync_enter((id)self);
@try{
  // Critical code goes here
}
@finally (id exception){
  objc_sync_exit((id)self);
}
...
\end{verbatim}
  \centering{}
  \caption{The \texttt{@synchronized} construct translated into run-time functions.}
  \label{fig:synchronized_translated}
\end{figure}

In order to avoid deadlocks in case of an exception, the critical section needs to be wrapped in \verb=@try-@finally=. No catching must be performed as the exception might need to be caught in a call above in the stack trace. Note that each object does not have its own lock (in the \'Etoil'e run-time it does, though), but a pool of locks is used instead.

The \verb=@try-@catch-@finally= is, of course, again translated into C code, to be precise, the compiler uses the \verb=_setjmp= function to install a stack exception data and the run-time uses \verb=longjmp= function to throw exceptions - this is, however, only used by the old run-time - Apple's new run-time uses the C++ \verb=unwind= library.

\subsection{Protocols}

Protocols are a way of declaring methods with no implementation that classes conforming to that particular protocol should implement. This mechanism is called \verb=interface= is Java, for example.

When a class conforms to a protocol, a pointer to a protocol is installed in its protocol list, which is quite obvious, but brings up a question, what is a protocol, in terms of structures.

Protocols are really nothing, but instances of an internal class \verb=Protocol=:

\begin{verbatim}
@protocol SomeProtocol
-(id)protocolMethod;
@end

...
Class cl = objc_getClass("Protocol");
BOOL isProtocol = [@protocol(SomeProtocol) isKindOfClass:cl];
...
\end{verbatim}

The \verb=isProtocol= variable is set to \verb=YES=. Or, another way:

\begin{verbatim}
printf("%s\n", class_getName(@protocol(SomeProtocol)->isa));
\end{verbatim}

This prints out, indeed, \verb=Protocol=. Interestingly enough, the

\begin{verbatim}@protocol(SomeProtocol)\end{verbatim}
  
construct is not simply translated to

\begin{verbatim}objc_getProtocol("SomeProtocol")\end{verbatim}

but is translated into 

\begin{verbatim}(Protocol *)&_OBJC_PROTOCOL_SomeProtocol\end{verbatim}

- a pointer to a certain exported protocol structure, that is part of the binary. And this is not an optimization because the protocol is declared in the same source file, the same mechanism applies when pointing to a protocol declared in an external framework.

Of course, the run-time needs to populate the \verb=isa= pointers of each protocol when loading the binary.

\subsection{Required Classes}

There are several language constructs that are compiled directly into objects, requiring the run-time to include classes for these objects.

\paragraph{Strings} The regular \verb=char*= C strings in quotes are simply a pointer to a chunk of memory that is typed as an array of \verb=char=s, which by definition cannot be an object, as the first field of \verb=id= needs to be an \verb=isa= pointer. Hence it cannot be treated as an object and has to be wrapped in an object representing an Objective-C string. Which is why the \verb=@"Objective-C String"= notation needs to be used.

This poses a question what is the string compiled to.

\begin{verbatim}
NSString *myString = @"My String";
\end{verbatim}

This gets compiled into:

\begin{verbatim}
NSString *myString = (NSString *)&__NSConstantStringImpl_test_m_0;
\end{verbatim}

Now, what is \verb=__NSConstantStringImpl_test_m_0=? That is actually a static object:

\begin{figure}[H]
\begin{verbatim}
static __NSConstantStringImpl 
        __NSConstantStringImpl_test_m_0 
        __attribute__ ((section ("__DATA, __cfstring"))) = 
         {
           __CFConstantStringClassReference,
           0x000007c8,
           "My String",
           9
         };
\end{verbatim}
  \centering{}
  \caption{An Objective-C string literal compiled into a static object structure.}
  \label{fig:NSString_compiled}
\end{figure}

This structure does indeed resemble an object - the first field is \newline{}\verb=__CFConstantStringClassReference=, which seems like something that could be the \verb=isa= pointer, followed by a hexadecimal number (flags), the actual \verb=char*= string and length of this string. And indeed:

\begin{verbatim}
struct __NSConstantStringImpl {
  int *isa;
  int flags;
  char *str;
  long length;
};
\end{verbatim}

Apple's implementation takes advantage of the CoreFoundation framework and toll-free bridging, a mechanism where some kinds of objects from the CoreFoundation framework may be used as Objective-C objects. This means that the run-time doesn't include a class implementing constant strings, thus forcing all Objective-C programs that wish to use literal strings to link against the CoreFoundation framework. The GCC implementation, on the other hand, includes a \verb=NXConstantString= class. Unlike Apple's implementation, the run-time then needs to fill in the \verb=isa= pointers with actual class pointers, however.

\paragraph{Blocks}

Apple has introduced lambda functions C language extension in OS X 10.6 - so called blocks:

\begin{verbatim}
void(^myBlock)(void *) = ^(void *arg){
  // Do something with the argument
};
\end{verbatim}

This declares an anonymous function that can be used even out of the current scope and the function itself can freely use variables from within the scope of the method where the block is declared in a read-only manner. For read-write access to variables, the variables need to be declared as \verb=__block=, which generally makes them static variables, allowing the block to modify the variable even when the program execution is already out of scope.

This simple block gets translated into this piece of code (some minor changes to the code have been made to improve readability):

\begin{figure}[H]
\begin{verbatim}
struct __block_impl {
  void *isa;
  int Flags;
  int Reserved;
  void *FuncPtr;
};


struct __myBlockImpl {
  struct __block_impl impl;
  struct __myBlockDescription *Desc;
  __myBlockImpl(void *fp, struct __myBlockDescription *desc, 
                                                int flags=0) {
    impl.isa = &_NSConcreteStackBlock;
    impl.Flags = flags;
    impl.FuncPtr = fp;
    Desc = desc;
  }
};

static void __myBlock_block_fnc(struct __myBlockImpl *__cself, 
                                                    void *arg){
  // Do something with the argument
}

static struct __myBlockDescription {
  unsigned long reserved;
  unsigned long Block_size;
} __myBlockDescription_DATA = { 0, sizeof(struct __myBlockImpl) };

...

void(*myBlock)(void *) = &__myBlockImpl(__myBlock_block_fnc, 
                                      &__myBlockDescription_DATA);

\end{verbatim}
  \centering{}
  \caption{Block declaration and usage declared.}
  \label{fig:block_compiled}
\end{figure}

While this indeed seems like a lot of 'fuzz' for a block that does nothing, the actual mechanism behind is a lot more intriguing once you start using variables from within the scope inside the block - then all the variables get copied over into \verb=__myBlockImpl= (that's why the \verb=Block_size= gets assigned \verb=sizeof(=\verb=struct= \verb=__myBlockImpl)= as the \verb=_myBlockImpl= can be as large as possible depending on the number of variables used within the block).

Apart from the C++ usage (the structure constructor), the noteworthy part is the \verb=impl= field of \verb=__myBlockImpl= - which begins with an \verb=isa= pointer which is (again) filled with a specific class pointer. This allows the structure to be sent ARR methods, and hence be treated as an object, allowing it to be added to the \verb=NSArray= and \verb=NSDictionary= collections.
